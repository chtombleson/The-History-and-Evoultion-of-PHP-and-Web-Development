\documentclass{book}
\usepackage{hyperref}
\usepackage{listings}
\usepackage{color}

\definecolor{mygreen}{rgb}{0,0.6,0}
\definecolor{mygray}{rgb}{0.5,0.5,0.5}
\definecolor{mymauve}{rgb}{0.58,0,0.82}

\lstset{ %
  backgroundcolor=\color{white},   % choose the background color; you must add \usepackage{color} or \usepackage{xcolor}
  basicstyle=\footnotesize,        % the size of the fonts that are used for the code
  breakatwhitespace=false,         % sets if automatic breaks should only happen at whitespace
  breaklines=true,                 % sets automatic line breaking
  captionpos=b,                    % sets the caption-position to bottom
  commentstyle=\color{mygreen},    % comment style
  deletekeywords={...},            % if you want to delete keywords from the given language
  escapeinside={\%*}{*)},          % if you want to add LaTeX within your code
  extendedchars=true,              % lets you use non-ASCII characters; for 8-bits encodings only, does not work with UTF-8
  frame=single,                    % adds a frame around the code
  keepspaces=true,                 % keeps spaces in text, useful for keeping indentation of code (possibly needs columns=flexible)
  keywordstyle=\color{blue},       % keyword style
  language=Perl,                   % the language of the code
  morekeywords={*,...},            % if you want to add more keywords to the set
  numbers=left,                    % where to put the line-numbers; possible values are (none, left, right)
  numbersep=5pt,                   % how far the line-numbers are from the code
  numberstyle=\tiny\color{mygray}, % the style that is used for the line-numbers
  rulecolor=\color{black},         % if not set, the frame-color may be changed on line-breaks within not-black text (e.g. comments (green here))
  showspaces=false,                % show spaces everywhere adding particular underscores; it overrides 'showstringspaces'
  showstringspaces=false,          % underline spaces within strings only
  showtabs=false,                  % show tabs within strings adding particular underscores
  stepnumber=2,                    % the step between two line-numbers. If it's 1, each line will be numbered
  stringstyle=\color{mymauve},     % string literal style
  tabsize=2,                       % sets default tabsize to 2 spaces
  title=\lstname                   % show the filename of files included with \lstinputlisting; also try caption instead of title
}

\setlength{\parindent}{0cm}

\begin{document}
\title{The History and Evoultion of PHP and Web Development}
\author{By Christopher Tombleson \href{mailto:chris@cribznetwork.com}{chris@cribznetwork.com}}
\date{\today}
\maketitle

\tableofcontents

\chapter{Intro}
\section{Preamble}
I will be exploring the evoultion of web development and PHP from it's humble beginnings as some C based cgi script though till today.

I will also explore some other languages that have been prevalent in the evoultion, expansion and dominance of the web in our everyday life.

I'm searching the web and asking people gathering information about the evoultion of web development but more so the evoultion of PHP.

\section{About the Author}

I've developing and writing web applications since 2008 mostly in PHP but I've work with Perl, Ruby and sadly that evil thing we call .NET

My intrigue for web development and programming come from me asking myself the question 'How does the web work? How can I make something cool? How does facebook work?'.

So I started the way many people do which is static HTML\ldots{}Which is cool for a week, then your like 'There has to be an easier way?'. Easier way === PHP, PHP 4 to be exact. Lets just say I don't miss those days.

Nowadays I'm an advide opensource developer and contributor, mostly to PHP projects. Although I'm knowledgable in many different languages such Perl, Ruby, Python, Javascript, C++, Lua\ldots{}The list goes on.

But for the purposes of this document I will be focusing mainly on PHP but will hit on some Perl (because it's awesome), Ruby, maybe some Python and some Javascript, node.js.

\chapter{The Web - Early Days}
From the days of the Gopher protocol and text based browsing to graphical interface and HTML/HyperText. The early 1980's and though to the mid 1990's the progression from terminal based text browsers to Netscape Navigator \& Internet Explorer (evil).

The World Wide Web (good old www) invented \& developed by Sir Tim Berners-Lee in 1989 at CERN and also created the worlds first website,
\href{http://info.cern.ch/hypertext/WWW/TheProject.html}{info.cern.ch}\cite{wiki-History-of-the-internet, cern}.

But as most of us know the web is nothing without a protocol to deliver the content, no worries the Sir Tim Berners-Lee also create the tools that really made the web such as the HTTP protocol (HyperText Transfer Protocol), HTML (HyperText Markup Language), the first web browser called World Wide Web and don't forget about the first HTTP server\cite{wiki-History-of-the-web}.

For more info about the world first website: \href{http://info.cern.ch/}{http://info.cern.ch/}.

If your like me your probably wondering how did we go from the worlds first website to the websites of today\ldots{}

\section{The Birth of CGI (Common Gateway Interface)}
The birth of CGI finally gave us away of run a program/script on a web server and then render the output of these scripts and send it back to the client.

The language of choice back when CGI was released in 1993\cite{royal-pingdom} was Perl.

\section{Perl on the web}
Perl is a widely used general purpose scripting language. Initial developed by Larry Wall in 1987 for scripting on Unix systems.

Perl 5 become popular in the late 1990's as a CGI scripting language. Perl 5 was released in late 1997, one thing that made Perl even popular for web development was the creation of CGI.pm.

CGI.pm is a Perl module that provides an api for creating CGI (Common Gateway Interface) web applications.

\subsection{Example Perl CGI Script}
\begin{lstlisting}
use CGI ':standard';
 
print header,
    start_html('A Simple CGI Page'),
    h1('A Simple CGI Page'),
    start_form,
    'Name: ',
    textfield('name'), br,
    'Age: ',
    textfield('age'), p,
    submit('Submit!'),
    end_form, p,
    hr;
 
print 'Your name is ', param('name'), br if param 'name';
print 'You are ', param('age'), ' years old.' if param 'age';
 
print end_html;

\end{lstlisting}

\section{Early PHP}
PHP was created in 1994 by Rasmus Lerdorf and was original a set of C CGI (Common Gateway Interface) binaries these were known as "Personal Home Page Tools" or "PHP Tools".

In 1995 Rasmus release the source for PHP Tools so it could be freely used and developed by others.


\chapter{The rise of server side languages in the web}
From 1993 with the release of the Common Gateway Interface (CGI) to the early 2000's many new programming languages most scripting languages were born such as PHP, Python, Ruby and also new versions of those Languages.

Here is a list of release years\cite{royal-pingdom}: 

\begin{itemize}
\item 1993 - Common Gateway Interface 
\item 1994 - Perl 5 (still going strong after 19 years) \& Python 1.0 
\item 1995 - PHP 2 (Known at the time as Personal Home Page), Ruby 
\item 1996 - ASP 1.0, Python 1.4 * 1998 - PHP 3, Java Servlet 2.1 
\item 1999 - Java Servlet 2.2 * 2000 - PHP 4, ASP 3.0, Python 2.0
\item 2001 - Java Servlet 2.3, Python 2.2 * 2002 - ASP.Net 1.0 (Evil)
\item 2003 - ASP.Net 1.1 (Still Evil), Java Servlet 2.4, Python 2.3 
\item 2004 - PHP 5, Ruby on Rails 0.5.0, Python 2.4 
\item 2005 - ASP.Net 2, Java Servlet 2.5, Python 2.5, Ruby on Rail 1.0.0 
\item 2006 - Ruby on Rails 1.1.0
\item 2007 - ASP.Net 3.5, Ruby on Rails 1.2.0
\end{itemize}

As you can see in the time line above the whole web exploded on to the scene in a big way in 1993.


\begin{thebibliography}{56}
\bibitem{wiki-History-of-the-internet}
\href{http://en.wikipedia.org/wiki/History_of_the_Internet}{Wikipedia - History of the Internet}

\bibitem{cern}
\href{http://info.cern.ch/}{Cern}

\bibitem{wiki-History-of-the-web}
\href{http://en.wikipedia.org/wiki/History_of_the_World_Wide_Web}{Wikipedia - History of the World Wide Web}

\bibitem{royal-pingdom}
\href{http://royal.pingdom.com/2007/12/07/a-history-of-the-dynamic-web/}{Royal Pingdom - A history of the Dynamic Web}

\bibitem{wiki-Perl-CGI}
\href{http://en.wikipedia.org/wiki/CGI.pm}{Wikipedia - Perl CGI.pm}

\bibitem{wiki-Perl}
\href{http://en.wikipedia.org/wiki/Perl}{Wikipedia - Perl}
\end{thebibliography}

\end{document}
